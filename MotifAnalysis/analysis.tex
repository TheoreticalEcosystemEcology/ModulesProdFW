\documentclass[11pt]{article}
\usepackage[sc]{mathpazo}
\usepackage[parfill]{parskip}

\title{Conceptual approach to the link between motif composition and functioning}
\author{T. Poisot}

\begin{document}
	
	\maketitle

	A motif is any set of three species having a degree of at least one. For
	simplicity we focus on the case of single-link degrees, of which there are
	five, are these are the most commonly occuring.

	We assume that all species have equal growth rate $r$ and crowding factor
	$q$, so that (i) they can persist at an equilibrium density higher than 0
	when alone, and (ii) the only differences in their equilibrium dynamics is
	triggered by interactions. For any species $i$ in a motif, its dynamics is
	given by

	\begin{equation}
		\frac{dN_i}{dt} = N_i\left[r-qN_i+\sum_j\left(\alpha_{ij}-\alpha_{ji}\right)N_j\right]
	\end{equation}

	To alternate between one of the five single-link motifs, one simply needs
	to switch the relevant $\alpha$ coefficients to 0. For example, in a
	linear food chain, all coefficients but $\alpha_{12}$ and $\alpha_{23}$
	are turned off.

	We derive the equilibrium density of this system with three species,
	yielding general expressions for $N_1^*$, $N_2^*$ and $N_3^*$. We furthermore
	define the total equilibrium biomass of the system as $T^* = \sum_iN_i^*$.
	We investigate the function of $T^*$ as a function of the values of all
	$\alpha$ coefficients.

	The general expression of $T^*$ is

	\begin{equation}
		T^* = r\times\frac{\alpha_{12}-\alpha_{13}-\alpha_{21}+\alpha_{23}+\alpha_{31}-\alpha_{32}+3q}{(\alpha_{12}-\alpha_{21})(\alpha_{12}-\alpha_{21}-\alpha_{23}-\alpha_{32})+q\times(q-\alpha_{13}+\alpha_{31})}
	\end{equation}

	Using this general expression, we can derive the equilibrium biomass
	$T_n^*$ in any motif $n$. To simplify this expression, we correct the
	total biomass by the species growth rate, so that $B_n^* = T_n^*/r$ In a
	linear food chain ($n=1$), all coefficients but $\alpha_{12}$ and
	$\alpha_{23}$ are set to 0, yielding

	\begin{equation}
		B_1^* = \frac{\alpha_{12}+\alpha_{23}+3q}{\alpha_{12}\times(\alpha_{12}-\alpha_{23})}
	\end{equation}

	In an omnivory motif ($n = 2$), species 1 can consume species 3, so that
	$\alpha_{13}$ is restored. This yields

	\begin{equation}
		B_2^* = \frac{\alpha_{12}-\alpha_{13}+\alpha_{23}+3q}{\alpha_{12}\times(\alpha_{12}-\alpha_{23})+q\times(q-\alpha_{13})}
	\end{equation}

	In a trophic loop motif ($n=3$), species 3 becomes a consumer of species
	1, so that $\alpha_{13}$ is set to 0, and $\alpha_{31}$ is restored. This
	yields

	\begin{equation}
		B_3^* = \frac{\alpha_{12}+\alpha_{23}+\alpha_{31}+3q}{\alpha_{12}\times(\alpha_{12}-\alpha_{23})+q\times(q+\alpha_{31})}
	\end{equation}

	In an exploitative competition motif ($n=4$), species 2 is consumed by
	both species 1 and 3, so that only $\alpha_{12}$ and $\alpha_{32}$ are
	kept. This yields

	\begin{equation}
		B_4^* = \frac{\alpha_{12}-\alpha_{32}+3q}{\alpha_{12}\times(\alpha_{12}-\alpha_{32})}
	\end{equation}

	Finally, in an apparent competition motif ($n=5$), species 2 is consuming
	both species 1 and 3, so that only $\alpha_{21}$ and $\alpha_{23}$ are
	kept. This yields

	\begin{equation}
		B_5^* = \frac{\alpha_{23}-\alpha_{21}+3q}{\alpha_{21}\times(\alpha_{21}+\alpha_{23})}
	\end{equation}

\end{document}